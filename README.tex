% Created 2022-03-25 sex 20:43
% Intended LaTeX compiler: pdflatex
\documentclass[11pt]{article}
\usepackage[utf8]{inputenc}
\usepackage[T1]{fontenc}
\usepackage{graphicx}
\usepackage{longtable}
\usepackage{wrapfig}
\usepackage{rotating}
\usepackage[normalem]{ulem}
\usepackage{amsmath}
\usepackage{amssymb}
\usepackage{capt-of}
\usepackage{hyperref}

\theoremstyle{definition}
\newtheorem{teo}{Teorema}[section]
\theoremstyle{definition}
\newtheorem{defi}{Definicao}[section]
\theoremstyle{remark}
\newtheorem{obs}{Observação}[section]
\theoremstyle{remark}
\newtheorem{lema}{Lema}[section]
\theoremstyle{remark}
\newtheorem{prop}{Propriedade}[section]
\theoremstyle{remark}
\newtheorem{coro}{Corolario}[section]
\theoremstyle{definition}
\newtheorem{prep}{Preposição}[section]
\author{Ieremies Vieira da Fonseca Romero}
\date{\today}
\title{MO824 - Atividade 1}
\hypersetup{
 pdfauthor={Ieremies Vieira da Fonseca Romero},
 pdftitle={MO824 - Atividade 1},
 pdfkeywords={},
 pdfsubject={},
 pdfcreator={Emacs 27.2 (Org mode 9.6)}, 
 pdflang={Portuguese}}
\usepackage{biblatex}
\addbibresource{~/arq/bib.bib}
\begin{document}

\maketitle
\tableofcontents


\section{Formulação}
\label{sec:orgc279aa8}
O objetivo é minimizar os custos totais. Nesse problema, podemos modelar o custo como \(\min z = \sum \limits_{j \in J} \sum \limits_{f \in F} \sum \limits_{l \in L} x_{p,l, f} * (p_{p,l,f} + t_{p,f,j})\), onde \(x_{p,f,l}\) é a quantidade de toneladas do produto \(p \in P\) produzido pela máquina \(l \in L\) na fábrica \(f \in F\).

De restrições, temos:
\begin{itemize}
\item Soma dos produtos p produzidos em todas as máquinas de todas as fábricas deve ser IGUAL/MENOR a demanda do cliente. \(\sum_{f \in F} \sum_{l \in l} \sum_{p \in P} x_{p,f,l} < D{j,p}\) para todo cliente \(j \in J\)
\item A quantidade de matéria prima gasta em cada máquina tem que ser menor que a quantidade disponível. \(\sum_{p \in P} \sum_{l \in L} x_{p,f,l} r_{m,p,l} < R_{m,f}\) para toda matéria prima \(m \in M\) em toda fábrica \(f \in F\).
\item A quantidade produzida tem que ser menor que a capacidade da máquina.
\(\sum_{p \in P} x_{p,f,l} < C_{f,l}\) para cada máquina \(l \in L\) em cada fábrica \(f \in F\)
\item A quantidade produzida deve ser positiva. \(x_{p,f,l} > 0\) para todo \(p \in P\) para todo \(f \in F\) para todo \(l \in L\).
\end{itemize}
\end{document}
